\documentclass[a4paper]{article}

\title{Linear Logic as a Programming Language}
\begin{document}
\maketitle

\section{Introduction}

Habeas is to be a language based directly on linear logic.
The language will have a construct for each of the derivation rules of LL.
The rules are treated below.

\section{General notes about the rules}

Because of the difference in semantics between commas occurring on the
left and on the right side of the turnstiles, we will here write the
right-hand ones as semicolons instead, in an attempt to reduce risk of
confusing ourselves.

\paragraph{Concepts.} Central concepts regarding how the rules of
linear logic are to be translated into programming language constructs
are: types, variables, and values. These are as known from other
programming languages.

As it turns out, in linear logic there is an additional useful concept
which we will call \emph{activation records}.  This term, too, is
known from other programming languages, but activation records are
perhaps more visible at the low level in linear logic.

\paragraph{How to read the rules.} A raw type rule typically looks like this:
$$
{\Sigma,A \vdash B;\Delta}\over{\Sigma,A' \vdash B';\Delta}
$$

The mathematical reading of this is, ``if we can prove ${\Sigma,A
  \vdash B;\Delta}$, then we can prove ${\Sigma,A' \vdash B';\Delta}$''.


The interpretation of this rule as a programming language construct is
somewhat different:
\begin{quote}
  ''Starting with a value of type $A'$ (and possibly some other values
  $\Sigma$), we get to a state where we have a value of type $A$ (and
  still $\Sigma$, unchanged). Then control passes to some sub-part of
  the construct; when this part is done, it has left an activation
  record of type $B$ as well as some other activation records
  $\Delta$. The construct then transforms this into an activation
  record of type $B'$ (and still $\Delta$, unchanged).''
\end{quote}

Thus, the control flow is from the left-hand side of the lower rule to
the left-hand side of the upper rule (or rules -- there may be more
than one; there may even be zero); then through the steps prescribed
by whatever proof tree(s) are there to prove the upper rule(s); then
from the right-hand side of the upper rule(s) down to the right-hand
side of the lower rule.

In general, there are a number of values available, and a number of
activation records are produced.

\paragraph{The cast.} Linear logic contains four binary connectives (type operators) and
one unary operator:
\begin{center}
  \begin{tabular}{c|l|l}
    \hline
    \hline
    Operator & Name & Approximate meaning\\
    \hline
    $\otimes$ & Times & Product type\\
    $\oplus$  & Plus  & Sum type\\
    $\&$      & With  & Choice type\\
    $|$       & Par   & Independent evaluation\\
    \hline
    $\lnot$   & Not   & Resource sink\\
    \hline
  \end{tabular}
\end{center}

Each of the four binary operators have a neutral element:
\begin{center}
  \begin{tabular}{c|c|l|l}
    \hline
    \hline
    Operator & Neutral & Name & Approximate meaning\\
             & element &&\\
    \hline
    $\otimes$ & $1$    & One    & Void/unit\\
    $\oplus$  & $0$    & Zero   & Impossible to create\\
    $\&$      & $\top$ & Top    & Impossible to destroy\\
    $|$       & $\bot$ & Bottom & The terminated activation record\\
    \hline
  \end{tabular}
\end{center}

\section{Linear Logic Rules and Habeas Language Constructs}

In the following we go through each of the derivation rules of linear
logic and see, what language construct each gives rise to.

\subsection{Rules for $\otimes$, the Product Type}
% If Γ,A,B,Δ⊢Θ, then Γ,A⊗B,Δ⊢Θ; conversely, if Γ⊢Δ,A and Λ⊢B,Θ, then Γ,Λ⊢Δ,A⊗B,Θ.

\subsubsection{Product introduction}
\paragraph{Rule:}
$$
{
 {\Sigma_1 \vdash A;\Delta_1}
 \qquad
 {\Sigma_2 \vdash B;\Delta_2}
}\over{
 \Sigma_1, \Sigma_2 \vdash A \otimes B ; \Delta_1 ; \Delta_2
}
$$

\paragraph{What goes on:} Two subexpressions create values of type $A$
and $B$, respectively. These two values are then combined into a value
of product type $A \otimes B$.

% Of the available values, some go to creating a value of type $A$ (and a set of activation records), while the rest go to creating a value of type $B$ (and another set of activation records). The two values are then combined into a value of product type $A \otimes B$ (and the union of the activation records).

\paragraph{Constraints and impact:} The available values are divided
between the two subexpressions.  Any extra activation records created
by the two subexpressions continue to exist.

\paragraph{Language construct:} Tuple creation, denoted by
\begin{quote}\texttt
  (tuple ($e_1$ $e_2$ $\ldots$ $e_n$))
\end{quote}
(Note that the language construct is $n$-ary rather than binary.)

\subsubsection{Product elimination}
\paragraph{Rule:}
$$
{
  \Sigma, A, B \vdash \Delta
}\over{
  \Sigma, A\otimes B \vdash \Delta
}
$$

\paragraph{What goes on:} A value of type $A \otimes B$ is split into two values of type $A$ and $B$, respectively.
\paragraph{Constraints and impact:} The value of type $A \otimes B$ is consumed; two the new values become available; all other available values continue to exist.
\paragraph{Language construct:} Tuple deconstruction, denoted by
\begin{quote}\texttt
  (untuple ($v_1$ $v_2$ $\ldots$ $v_n$) $e_{tuple}$ $e_{body}$)
\end{quote}
(Note that the language construct is $n$-ary rather than binary.)\\
The variables $v_i$ are bound to the elements of the tuple, and are available in the expression $e_{body}$.


\subsection{Rules for $\oplus$, the Sum Type}
\subsection{Rules for $\&$, the Choice Type}
\subsection{Rules for $|$, the Par Type}

\paragraph{Rule:}
\paragraph{What goes on:}
\paragraph{Constraints and impact:}
\paragraph{Language construct:}



% 'par' + negation provides a function/call mechanism.
% 'with' + negation provides a multi-return-path mechanism.





\end{document}
