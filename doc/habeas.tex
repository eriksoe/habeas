\documentclass[a4paper]{article}

\title{Linear Logic as a Programming Language}
\begin{document}
\maketitle

\section{Introduction}

Habeas is to be a language based directly on linear logic.
The language will have a construct for each of the derivation rules of LL.

The design aim is to reflect linear logic as faithfully as possible,
with no arbitrary restrictions, while at the same time packaging it in
a useful manner.

The rules are treated below, one by one.

\section{General notes about the rules}

\paragraph{Notation.}
As is the norm, $A, B, \ldots$ denote single
propositions/types/resources, while $\Gamma, \Delta, \ldots$ denote
(multi)sets of such.

Because of the difference in semantics between commas occurring on the
left and on the right side of the turnstiles, we will here write the
right-hand ones as semicolons instead, in an attempt to reduce risk of
confusing ourselves.

For typographical reasons, we will use $A|B$ rather than the more
common ``inverted ampersand'' operator; both are used elsewhere.
Less commonly, we use $\lnot A$ rather than $A^\bot$.

\paragraph{Concepts.} Central concepts regarding how the rules of
linear logic are to be translated into programming language constructs
are: types, variables, and values. These are as known from other
programming languages.
Variables are (static) names for the (dynamic) values available at a
given program point; to these variables can (statically) be assigned
types which describe which kinds of values might occur.

\paragraph{Channels.} As it turns out, in linear logic there is an additional useful concept
which we will call \emph{channels}.
Whereas in e.g. lambda calculus, an expression returns one value of a
certain types, in linear logic we can interpret expressions as
returning a value of a certain type \emph{plus} a set of channels
(also typed).

% TODO: The semantics for channels may fit well to something like catch handlers, or shell redirections: they are set up on the way in, rather than dealt with on the way out.

\paragraph{How to read the rules.} A raw type rule typically looks like this:
$$
{\Gamma,A \vdash B;\Delta}\over{\Gamma,A' \vdash B';\Delta}
$$

The mathematical reading of this is, ``if we can prove ${\Gamma,A
  \vdash B;\Delta}$, then we can prove ${\Gamma,A' \vdash B';\Delta}$''.


The interpretation of this rule as a programming language construct is
somewhat different:
\begin{quote}
  ''Starting with a value of type $A'$ (and possibly some other values
  $\Gamma$), we get to a state where we have a value of type $A$ (and
  still $\Gamma$, unchanged). Then control passes to some sub-part of
  the construct; when this part is done and control return to the construct,
  it has left a result of type $B$ as well as a set of channels
  $\Delta$. The construct then transforms this into a value of type
  $B'$ (and still $\Delta$, unchanged).''
\end{quote}

Thus, the control flow up-right-down:
from the left-hand side of the lower rule to
the left-hand side of the upper rule (or rules -- there may be more
than one; there may even be zero); then through the steps prescribed
by whatever proof tree(s) are there to prove the upper rule(s); then
from the right-hand side of the upper rule(s) down to the right-hand
side of the lower rule.

In general, there are a number of values available, and a value plus a
number of channels are produced.

Note that our choice to interpret the right-hand side as a return
value plus a number of channels is rather arbitrary: there is no a
priori difference between the return value and a channel.

\paragraph{The cast.} Linear logic contains four binary connectives (type operators) and
one unary operator:
\begin{center}
  \begin{tabular}{c|l|l}
    \hline
    \hline
    Operator & Name & Approximate meaning\\
    \hline
    $\otimes$ & Times & Product type\\
    $\oplus$  & Plus  & Sum type\\
    $\&$      & With  & Choice type\\
    $|$       & Par   & Independent evaluation\\
    \hline
    $\lnot$   & Not   & Resource sink\\
    \hline
  \end{tabular}
\end{center}

Each of the four binary operators have a neutral element:
\begin{center}
  \begin{tabular}{c|c|l|l}
    \hline
    \hline
    Operator & Neutral & Name & Approximate meaning\\
             & element &&\\
    \hline
    $\otimes$ & $1$    & One    & Void/unit\\
    $\oplus$  & $0$    & Zero   & Impossible to create\\
    $\&$      & $\top$ & Top    & Impossible to destroy\\
    $|$       & $\bot$ & Bottom & A terminated process\\
    \hline
  \end{tabular}
\end{center}


\section{Linear Logic Rules and Habeas Language Constructs}

In the following we go through each of the derivation rules of linear
logic and see, what language construct each gives rise to.

\paragraph{Design choice: Symbolic names.}
Note that whereas the rules of linear logic deal with binary
operators, we have here chosen to operate with $n$-ary type
constructors in most cases.  Also, in most cases we have elected to
let the constituents be addressed by symbolic names (\emph{tags}),
rather than merely by their position.

Thus, instead of for example having a binary sum type $A \oplus B$
with ``left-'' and ``right-''injection functions, we have an $n$-ary
sum type $(\mathit{tag}_1:T_1 \oplus \mathit{tag}_2:T_2 \oplus \ldots
\oplus \mathit{tag}_n:T_n)$, with an injection function for each tag.


\subsection{General rules}

\subsubsection{The ``cut'' rule}
\paragraph{Rule:}
$$
{
  \Gamma_1 \vdash A; \Delta_1
  \qquad
  \Gamma_2,A \vdash \Delta_2
}\over{
  \Gamma_1, \Gamma_2 \vdash \Delta_1; \Delta_2
}
$$

\paragraph{What goes on:} First, one subexpression evaluates to a
value of type $A$, then another subexpression consumes that values.

\paragraph{Constraints and impact:} The entire expression consumes the
union of what the two subexpressions consume, save for the mentioned
value; and it produces a set of channels which is the union of that
produced by the two subexpressions. The result of the entire
expression is that of the second subexpression.

\paragraph{Language construct:}
\begin{quote}\tt
(let (($v$ $e_{bind}$)) $e_{body}$)
\end{quote}
This is generalized to the \texttt{let} and \texttt{let*} constructs
known from Lisp:
\begin{quote}\tt
(let\\
  \phantom{ }\quad (($v_1$ $e_{bind,1}$)\\
  \phantom{ }\quad \ldots\\
  \phantom{ }\quad\ ($v_n$ $e_{bind,n}$))\\
  \phantom{ }\quad $e_{body}$)
\end{quote}
defined as \texttt{(let (($v_1$ $e_{bind,1}$)) (let (($v_2$ $e_{bind,2}$)) \ldots\ $e_{body}$)\ldots)}.

(The difference between \texttt{let} and \texttt{let*} is in whether
$e_{bind,j}$ may refer to $v_i$ for $i<j$. \texttt{let*} allows this,
while \texttt{let} does not.)



\subsection{Rules for $\otimes$, the Product Type}
% If Γ,A,B,Δ⊢Θ, then Γ,A⊗B,Δ⊢Θ; conversely, if Γ⊢Δ,A and Λ⊢B,Θ, then Γ,Λ⊢Δ,A⊗B,Θ.

(Note that the language's product type is $n$-ary rather than binary.)

\subsubsection{Product introduction}
\paragraph{Rule:}
$$
{
 {\Gamma_1 \vdash A;\Delta_1}
 \qquad
 {\Gamma_2 \vdash B;\Delta_2}
}\over{
 \Gamma_1, \Gamma_2 \vdash A \otimes B ; \Delta_1 ; \Delta_2
}
$$

\paragraph{What goes on:} Two subexpressions create values of type $A$
and $B$, respectively. These two values are then combined into a value
of product type $A \otimes B$.

% Of the available values, some go to creating a value of type $A$ (and a set of channels), while the rest go to creating a value of type $B$ (and another set of channels). The two values are then combined into a value of product type $A \otimes B$ (and the union of the channels).

\paragraph{Constraints and impact:} The available values are divided
between the two subexpressions.  Any channels created
by the two subexpressions continue to exist.

\paragraph{Language construct:} Tuple creation, denoted by
\begin{quote}\tt
  (tuple ($e_1$ $e_2$ $\ldots$ $e_n$))
\end{quote}

\subsubsection{Product elimination}
\paragraph{Rule:}
$$
{
  \Gamma, A, B \vdash \Delta
}\over{
  \Gamma, A\otimes B \vdash \Delta
}
$$

\paragraph{What goes on:} A value of type $A \otimes B$ is split into two values of type $A$ and $B$, respectively.

\paragraph{Constraints and impact:} The value of type $A \otimes B$ is consumed; the two new values become available; all other available values continue to exist.

\paragraph{Language construct:} Tuple deconstruction, denoted by
\begin{quote}\tt
  (untuple ($v_1$ $v_2$ $\ldots$ $v_n$) $e_{tuple}$ $e_{body}$)
\end{quote}
The variables $v_i$ are bound to the elements of the tuple, and are available in the expression $e_{body}$.



\subsection{Rules for $\oplus$, the Sum Type}
% Dually, if Γ⊢Δ,A,Θ or Γ⊢Δ,B,Θ, then Γ⊢Δ,A⊕B,Θ; conversely, if Γ,A,Δ⊢Θ and Γ,B,Δ⊢Θ, then Γ,A⊕B,Δ⊢Θ.

(Note that the language construct is $n$-ary rather than binary, and
uses symbolic tags.)


\subsubsection{Sum introduction}
\paragraph{Rule:}
$$
{{
  \Gamma \vdash A;\Delta
}\over{
  \Gamma \vdash A \oplus B;\Delta
}}
\qquad \textrm{and} \qquad
{{
  \Gamma \vdash B;\Delta
}\over{
  \Gamma \vdash A \oplus B;\Delta
}}
$$

\paragraph{What goes on:} A value of type $A$ or of type $B$ is
converted into a value of type $A \oplus B$.

\paragraph{Constraints and impact:} The value of type $A$ or $B$ is
consumed; the new value becomes available; all other available values
continue to exist.

\paragraph{Language construct:} Sumtype injection, denoted by
\begin{quote}\tt
  (inject $\mathit{tag}$ $e$)
\end{quote}


\subsubsection{Sum elimination}
\paragraph{Rule:}
$$
{
% if Γ,A,Δ⊢Θ and Γ,B,Δ⊢Θ, then Γ,A⊕B,Δ⊢Θ.
  \Gamma,A \vdash \Delta \qquad \Gamma,B \vdash \Delta
}\over{
  \Gamma, A\oplus B \vdash \Delta
}
$$

\paragraph{What goes on:} A value of type $A \oplus B$ is converted to
a value of type either $A$ or $B$. That value is then handled of one
of two subexpressions -- one for $A$ and one for $B$.

\paragraph{Constraints and impact:} The value of type $A \oplus B$ is
consumed; the value of type $A$ or type $B$ becomes available.
All other values remain available regardless of which subexpression is chosen.

Furthermore, the results and channel sets produced by the two
subexpressions must have the same types.
This is also the result and channel set produced by the entire
expression.

\paragraph{Language construct:}
\begin{quote}\tt
  (case $e$\\
  \phantom{ }\quad ($\mathit{tag_1}$ $\mathit{e_1}$)\\
  \phantom{ }\quad \ldots\\
  \phantom{ }\quad ($\mathit{tag_n}$ $\mathit{e_n}$))
\end{quote}



\subsection{Rules for $\&$, the Choice Type}
% If Γ⊢Δ,A,Θ and Γ⊢Δ,B,Θ, then Γ⊢Δ,A&B,Θ.

(Note that the language construct is $n$-ary rather than binary, and
uses symbolic tags.)

\subsubsection{Choice introduction}

\paragraph{Rule:}
$$
{
  \Gamma \vdash A; \Delta
  \qquad
  \Gamma \vdash B; \Delta
}\over{
  \Gamma \vdash A\&B; \Delta
}
$$
\paragraph{What goes on:} Two subexpressions which differ in that one
produces a value of type $A$ and the other produces a value of type
$B$ are combined into a value of type $A\&B$.

\paragraph{Constraints and impact:} The two subexpressions must not
otherwise differ in which values they consume or in the number and
types of channels they produce.

\paragraph{Language construct:}
\begin{quote}\tt
  (choice\\
  \phantom{ }\quad ($\mathit{tag}_1$ $e_1$)\\
  \phantom{ }\quad \ldots\\
  \phantom{ }\quad ($\mathit{tag}_n$ $e_n$))
\end{quote}
The evaluation of the subexpressions themselves is postponed until the
choice is made; nor will the produced channels yield output until that time.


\subsubsection{Choice elimination}

% If Γ,A,Δ⊢Θ or Γ,B,Δ⊢Θ, then Γ,A&B,Δ⊢Θ

\paragraph{Rule:}
$$
{{
  \Gamma, A \vdash \Delta
}\over{
  \Gamma, A \& B \vdash \Delta
}}
\qquad \textrm{and} \qquad
{{
  \Gamma, B \vdash \Delta
}\over{
  \Gamma, A \& B \vdash \Delta
}}
$$

\paragraph{What goes on:} A value of a choice type is transformed into
a value of one of the types it contains.

\paragraph{Constraints and impact:}
From a local view, nothing else happens;
however, the choice may have the side effect of enabling output from
some channels. See the introduction rule.

\paragraph{Language construct:}
\begin{quote}\tt
(select $\mathit{tag}$ $e$)
\end{quote}


\subsection{Rules for $\lnot$, the Channel Sink Type}

Channels are non-local means of transfering values.
Each channel transfers one value.
A channel has two ends: the \emph{sink end}, into which the value is
put, and a \emph{source end}, from which the value can then be
extracted.

% If Γ⊢A,Δ, then Γ,A⊥⊢Δ; conversely and dually, if Γ,A⊢Δ, then Γ⊢A⊥,Δ.
\subsubsection{Channel sink introduction}

\paragraph{Rule:}
$$
{
  \Gamma, A \vdash \Delta
}\over{
  \Gamma \vdash \lnot A; \Delta
}
$$

\paragraph{What goes on:} A channel is introduced, in the form of a
resource source/sink pair. A subexpression can then operate as if a
resource of type $A$ is available.

\paragraph{Constraints and impact:} The resource sink is introduced as
an activation record of its own. The activation records resulting from
the subexpression are seen as immediately available.

\paragraph{Language construct:}
There are at least two choices here, depending on which channel is
chosen as the main output (the obvious choices being $\lnot A$ and the
result of the subexpression).
%
We'll choose a construct which is compatible with strict evaluation:
\begin{quote}\tt
  (channel-into $v$ $e_{body}$ $\mathit{tag}_{output}$)
\end{quote}
Here, $v$ is bound to the source end of the channel and has scope of
$e_{body}$.  The sink end of the channel is returned as the result of
the entire expression.
The result of the subexpression is returned as a new channel named
$\mathit{tag}_{output}$, along with the ones produced by $e_{body}$.

\subsubsection{Channel sink elimination}
\paragraph{Rule:}
$$
{
  \Gamma \vdash A; \Delta
}\over{
  \Gamma, \lnot A \vdash \Delta
}
$$

\paragraph{What goes on:} A channel sink is eliminated by feeding it
a value of the appropriate type.

\paragraph{Constraints and impact:} The channel sink disappears, along
with the value.

\paragraph{Language construct:}
\begin{quote}\tt
  (feed-channel $e_{sink}$ $e$)
\end{quote}
This construct evaluates to $\bot$ (the terminated channel value).


\subsection{Rules for $|$, the Par Type}

% If Γ⊢Δ,A,B,Θ, then Γ⊢Δ,A|B,Θ; conversely, if Γ,A⊢Δ and B,Θ⊢Λ, then Γ,A|B,Θ⊢Δ,Λ.

\subsubsection{Par introduction}

\paragraph{Rule:}
$$
{
  \Gamma \vdash A; B; \Delta
}\over{
  \Gamma \vdash A|B; \Delta
}
$$
\paragraph{What goes on:} Two channels are combined in a value of ``par'' type.
\paragraph{Constraints and impact:}
\paragraph{Language construct:}
\begin{quote}\tt
\end{quote}

\subsubsection{Par elimination}

\paragraph{Rule:}
$$
{
  \Gamma_1, A \vdash \Delta_1
  \qquad
  \Gamma_2, B \vdash \Delta_2
}\over{
  \Gamma_1, \Gamma_2, A|B \vdash \Delta_1; \Delta_2
}
$$

\paragraph{What goes on:} A value of ``par'' type is split into its
constituent channels, the outputs of which are each made available in
a subexpression. The subexpressions are then evaluated independently
(possibly in parallel; note that the channels may be triggered in any order.).

\paragraph{Constraints and impact:}
\paragraph{Language construct:}
\begin{quote}\tt
\end{quote}


\subsection{Rules for $!$, the Factory Type}

\subsubsection{Factory introduction}

%if Γ⊢Δ,A,Θ, then Γ⊢Δ,!A,Θ, whenever Γ consists entirely of propositions of the form !− while Δ and Θ consist entirely of propositions of the form ?−.

\paragraph{Rule:}
$$
{
  !\Gamma \vdash A; ?\Delta
}\over{
  !\Gamma \vdash !A; ?\Delta
}
$$
\paragraph{What goes on:} A subexpression is turned into a factory.

\paragraph{Constraints and impact:} This can only be done when no
linear resources are consumed or produced by the subexpression;
rather, all consumed resources must be factories themselves, while all
produced channels must be pollutions.

\paragraph{Language construct:}
\begin{quote}\tt
  (make-factory $e$)
\end{quote}

\subsubsection{Factory splitting}
% If Γ,!A,!A,Δ⊢Θ, then Γ,!A,Δ⊢Θ
\paragraph{Rule:}
$$
{
  \Gamma, !A, !A \vdash \Delta
}\over{
  \Gamma, !A \vdash \Delta
}
$$
\paragraph{What goes on:} A factory is split into two.
\paragraph{Constraints and impact:} (None.)
\paragraph{Language construct:}
\begin{quote}\tt
  (split-factory $v_1$ $v_2$ $e$)
\end{quote}

\subsubsection{Factory elimination (through use)}
% If Γ,A,Δ⊢Θ, then Γ,!A,Δ⊢Θ.
\paragraph{Rule:}
$$
{
  \Gamma, A \vdash \Delta
}\over{
  \Gamma, !A \vdash \Delta
}
$$
\paragraph{What goes on:} A factory is used for generating a value.
\paragraph{Constraints and impact:} (None.)
\paragraph{Language construct:}
\begin{quote}\tt
  (instantiate $e$)
\end{quote}

\subsubsection{Factory elimination (through disposal)}
% If Γ,Δ⊢Θ, then Γ,!A,Δ⊢Θ, for any A.
\paragraph{Rule:}
$$
{
  \Gamma \vdash \Delta
}\over{
  \Gamma, !A \vdash \Delta
}
$$
\paragraph{What goes on:} A factory is disposed of.
\paragraph{Constraints and impact:} (None.)
\paragraph{Language construct:}
\begin{quote}\tt
  (dispose-factory $e$)
\end{quote}
This expression evaluates to a value of the unit type $1$.

\subsection{Rules for $?$, the Pollution Type}

\subsubsection{Pollution introduction (empty case)}
% If Γ⊢Δ,Θ, then Γ⊢Δ,?A,Θ for any A.
\paragraph{Rule:}
$$
{
  \Gamma \vdash \Delta
}\over{
  \Gamma \vdash ?A;\Delta
}
$$
\paragraph{What goes on:} An empty pollution is created.
\paragraph{Constraints and impact:} (None.)
\paragraph{Language construct:}
\begin{quote}\tt
  (empty-pollution)
\end{quote}

\subsubsection{Pollution introduction (singleton case)}
% If Γ⊢Δ,A,Θ, then Γ⊢Δ,?A,Θ.
\paragraph{Rule:}
$$
{
  \Gamma \vdash A;\Delta
}\over{
  \Gamma \vdash ?A;\Delta
}
$$
\paragraph{What goes on:} A pollution is created from a single element.
\paragraph{Constraints and impact:} (None.)
\paragraph{Language construct:}
\begin{quote}\tt
  (single-pollution $e$)
\end{quote}

\subsubsection{Pollution combination}
% If Γ⊢Δ,?A,?A,Θ, then Γ⊢Δ,?A,Θ.
\paragraph{Rule:}
$$
{
  \Gamma \vdash ?A; ?A; \Delta
}\over{
  \Gamma \vdash ?A; \Delta
}
$$
\paragraph{What goes on:} Two pollutions of the same type are combined into one.
\paragraph{Constraints and impact:} (None.)
\paragraph{Language construct:}
\begin{quote}\tt
  (combine-pollution-with-channel $e$ $\mathit{tag}_{channel}$)
\end{quote}
% TODO: There is a channels vs. variables issue, generally: should it be possible to bind a set of channels?

\subsubsection{Pollution elimination}
% If Γ,A,Δ⊢Θ, then Γ,?A,Δ⊢Θ, whenever Γ and Δ consist entirely of propositions of the form !− while Θ consists entirely of propositions of the form ?−.
\paragraph{Rule:}
$$
{
  !\Gamma, A \vdash ?\Delta
}\over{
  !\Gamma, ?A \vdash ?\Delta
}
$$
\paragraph{What goes on:} A pollution value is consumed by a subexpression.

\paragraph{Constraints and impact:} The subexpression must not consume
or produce linear values; rather, all consumed resources must be
factories, while all produced channels must be also be pollutions.

\paragraph{Language construct:}
\begin{quote}\tt
  (eliminate-pollution $e_{pollution}$ $v$ $e_{elim}$)
\end{quote}
were $v$ has scope in $e_{elim}$.

% \paragraph{Rule:}
% $$
% {
% }\over{
% }
% $$
% \paragraph{What goes on:}
% \paragraph{Constraints and impact:}
% \paragraph{Language construct:}
% \begin{quote}\tt
% \end{quote}


% TODO: Remember rules for 'void' (1) and 'fail' ($\top$), and an
% implicit one for $\bot$.
% TODO: A construct for selecting a different channel (and naming the default one)


% 'par' + negation provides a function/call mechanism.
% 'with' + negation provides a multi-return-path mechanism.





\end{document}
